% vim:tw=72 sw=2 ft=tex
%         File: thoughts_report.tex
% Date Created: 2013 Jun 18
%  Last Change: 2014 Apr 23
%       Author: hhiker
\documentclass[a4paper]{article}
\usepackage{xltxtra}
\usepackage{xcolor}
\usepackage{xeCJK}
\usepackage{minted}
\usepackage{booktabs}
\usepackage{amsmath}
\usepackage{paralist}
\newcommand{\head}[1]{\textbf{#1}}
\setCJKmainfont[BoldFont=SimHei,ItalicFont=SimSun]{KaiTi}

\newtheorem{theorem}{Theorem}[section]
\newtheorem{lemma}[theorem]{Lemma}
\newtheorem{proposition}[theorem]{Proposition}
\newtheorem{corollary}[theorem]{Corollary}

\newenvironment{proof}[1][Proof]{\begin{trivlist}
\item[\hskip \labelsep {\bfseries #1}]}{\end{trivlist}}
\newenvironment{definition}[1][Definition]{\begin{trivlist}
\item[\hskip \labelsep {\bfseries #1}]}{\end{trivlist}}
\newenvironment{example}[1][Example]{\begin{trivlist}
\item[\hskip \labelsep {\bfseries #1}]}{\end{trivlist}}
\newenvironment{remark}[1][Remark]{\begin{trivlist}
\item[\hskip \labelsep {\bfseries #1}]}{\end{trivlist}}

\newcommand{\qed}{\nobreak \ifvmode \relax \else
      \ifdim\lastskip<1.5em \hskip-\lastskip
      \hskip1.5em plus0em minus0.5em \fi \nobreak
      \vrule height0.75em width0.5em depth0.25em\fi}

% There is a known bug:
% When both the amsmath and hyperref packages are loaded at the same time,
% the cleveref cross-referencing commands do not work when used within
% section titles. If anyone can figure out why, let me know! As a work-around,
% use \ref within section titles when your document uses both amsmath and
% hyperref.
% So hyperref package is disabled.
% \usepackage[colorlinks=true,linkcolor=red]{hyperref}
\usepackage{varioref}
\usepackage{cleveref}

\title{Proofs of Theorems about Trace of Matrix}
\author{Shuai Li}

\begin{document}
\maketitle
\tableofcontents
\pagebreak

\section{Definitions}

\begin{definition}
  The uppercase letters below all denote matrices and the trace of X is
  denoted $trX$.
\end{definition}

\begin{definition}
	Let f(A), where A is a matrix, be a function $\in R^{m\times n} \to
	R $, where
	\begin{displaymath}
		\nabla_Af(A) =
		\begin{pmatrix}
			\frac{\partial f}{\partial a_{11}} & \cdots & \frac{\partial f}{\partial a_{1n}}\\
			\vdots &  & \vdots\\
			\frac{\partial f}{\partial a_{m1}} & \cdots & \frac{\partial f}{\partial a_{mn}}
		\end{pmatrix}
\end{displaymath}

\end{definition}

\section{Theorems Summary}

The prerequisite of the following theorems is that matrices
multiplication makes sense. For example, $AB$ and $BA$ should both be
square matrices. $ABC$, $CAB$, $BCA$ should all be square matrices. But
they may not be of the same dimension.

\begin{enumerate}
	\item Commutative Law:

		$trAB = trBA$  

		$trABC = trCAB = trBCA$
	\item Suppose $f(A) = trAB$, then $\nabla_A trAB = B^T$.
	\item If $a \in R$, $tra = a$.
	\item $\nabla_A trABA^TC = CAB + C^TAB^T$
\end{enumerate}



\section{Theorems and its Proof}

\begin{theorem}
	trAB = trBA, where we suppose that $A \in R^{m \times n}, B \in R^{n
    \times m}$.
\end{theorem}

\begin{corollary}
	trABC = trCAB = trBCA
\end{corollary}

\begin{proof}
	First write the expressions of two sides of the equations:


	\begin{equation}
		trAB = \sum\limits^{m}_{i=1}\sum\limits^{n}_{j=1}a_{ij}b_{ji}
		\label{eq:trace_of_AB}
	\end{equation}

	\begin{equation}
		trBA = \sum\limits^{n}_{i=1}\sum\limits^{m}_{j=1}b_{ij}a_{ji}
		\label{eq:trace_of_BA}
	\end{equation}

	Switch the role of $i$ and $j$ of \cref{eq:trace_of_BA}, we get:
	\begin{displaymath}
		trBA = \sum\limits^{m}_{j=1}\sum\limits^{n}_{i=1}b_{ji}a_{ij}
	\end{displaymath}
	Change the summation sequence and exchange the position of $a$ and
	$b$, we have:
	\begin{displaymath}
		trBA = \sum\limits^{m}_{i=1}\sum\limits^{n}_{j=1}a_{ij}b_{ji}
	\end{displaymath}
	This is exactly the same with \cref{eq:trace_of_AB}.

	Proof is done.\qed
\end{proof}

\begin{theorem}
	Suppose $f(A) = trAB$, then $\nabla_A trAB = B^T$ where $A \in R^{m
    \times n}, B \in R^{n \times m}$.
	\label{thm:trAB_=_B^T}
\end{theorem}

\begin{proof}
	From \cref{eq:trace_of_AB}, we have:
	\begin{eqnarray}
		\nabla_AtrAB & = &
		\begin{pmatrix}
			\frac{\partial \sum\limits^{m}_{i=1}\sum\limits^{n}_{j=1}a_{ij}b_{ji}}{\partial a_{11}} & \cdots & \frac{\partial \sum\limits^{m}_{i=1}\sum\limits^{n}_{j=1}a_{ij}b_{ji}}{\partial a_{1n}}\\
			\vdots &  & \vdots\\
			\frac{\partial \sum\limits^{m}_{i=1}\sum\limits^{n}_{j=1}a_{ij}b_{ji}}{\partial a_{m1}} & \cdots & \frac{\partial \sum\limits^{m}_{i=1}\sum\limits^{n}_{j=1}a_{ij}b_{ji}}{\partial a_{mn}}
		\end{pmatrix}
		\label{eq:nabla_trace_AB}
	\end{eqnarray}

	For $(\nabla_AtrAB)_{pq} = \frac{\partial
	\sum\limits^{m}_{i=1}\sum\limits^{n}_{j=1}a_{ij}b_{ji}}{\partial
	a_{pq}}$, there is only one term contains $a_{pq}$,
	thus $(\nabla_AtrAB)_{pq} = b_{qp}$.

	Now, it is easy to see $\nabla_AtrAB = B^T$\qed

\end{proof}

\begin{theorem}
	If $a \in R$, $tra = a$
\end{theorem}

\begin{proof}
	This one is obvious.\qed
\end{proof}

\begin{theorem}
	$\nabla_A trABA^TC = CAB + C^TAB^T$, where $A \in R^{m \times n}, B
  \in R^{n \times n}, C \in R^{m \times m}$.
\end{theorem}

\begin{proof}
	At first, this one seems to be an application of
	\cref{thm:trAB_=_B^T}. However, since $A^T$ is contained in $BA^TC$,
	this is not true.

	But the good news is, we can do similar steps to prove this theorem.

	We denote $d_{ij}$ as the element at i row and j column of $BA^TC$.

	Then, for $trABA^TC$, we have
	\begin{displaymath}
		trABA^TC = \sum\limits^{m}_{i=1}\sum\limits^{n}_{j=1}a_{ij}d_{ji}
	\end{displaymath}
	Then, for $\nabla_A trABA^TC$, we have
	\begin{displaymath}
		\nabla_A trABA^TC = \nabla_A \sum\limits^{m}_{i=1}\sum\limits^{n}_{j=1}a_{ij}d_{ji}
	\end{displaymath}

	\begin{displaymath}
		(\nabla_A trABA^TC)_{pq} = d_{qp} +
		\sum\limits^{m}_{i=1}\sum\limits^{n}_{j=1}a_{ij}(\nabla_A
		d_{ji})_{pq}
	\end{displaymath}

	The first term is actually $(BA^TC)_{qp} = (C^TAB^T)_{pq}$

	The second term is harder to see.

	Let's first figure out what $d_{ji}$ is.

	\begin{eqnarray}
		d_{ji}	& = & (BA^TC)_{ji} \nonumber\\
		& = &
		\sum\limits^{n}_{k=1}b_{jk}(\sum\limits^{n}_{l=1}a_{lk}c_{li})
		\nonumber
	\end{eqnarray}

	Since only terms contain $a_{pq}$ matter, the second term become
	this:
	\begin{eqnarray}
		\sum\limits^{m}_{i=1}\sum\limits^{n}_{j=1}a_{ij}(\nabla_A
		d_{ji})_{pq} & = &
				\sum\limits^{m}_{i=1}\sum\limits^{n}_{j=1}a_{ij}b_{jq}c_{pi}
				\nonumber\\
				& = & \sum\limits^{m}_{i=1}\sum\limits^{n}_{j=1}c_{pi}a_{ij}b_{jq}
				\nonumber\\
				& = & (CAB)_{pq} \nonumber
	\end{eqnarray}

	Combine first and second term, we have:

	\begin{displaymath}
		\nabla_A trABA^TC = CAB + C^TAB^T
	\end{displaymath}
	\qed

\end{proof}


\end{document}
