% vim:tw=72 sw=2 ft=tex
%         File: thoughts_report.tex
% Date Created: 2013 Jun 18
%  Last Change: 2013 Dec 16
%       Author: hhiker
\documentclass[a4paper]{article}
\usepackage{xltxtra}
\usepackage{xcolor}
\usepackage{xeCJK}
\usepackage{minted}
\usepackage{booktabs}
\usepackage{amsmath}
\usepackage{paralist}
\usepackage[colorlinks=true,linkcolor=red]{hyperref}
% \usepackage{hyperref}
\usepackage{varioref}
\usepackage{cleveref}
\newcommand{\head}[1]{\textbf{#1}}
\setCJKmainfont[BoldFont=SimHei,ItalicFont=SimSun]{KaiTi}

\newtheorem{theorem}{Theorem}[section]
\newtheorem{lemma}[theorem]{Lemma}
\newtheorem{proposition}[theorem]{Proposition}
\newtheorem{corollary}[theorem]{Corollary}

\newenvironment{proof}[1][Proof]{\begin{trivlist}
\item[\hskip \labelsep {\bfseries #1}]}{\end{trivlist}}
\newenvironment{definition}[1][Definition]{\begin{trivlist}
\item[\hskip \labelsep {\bfseries #1}]}{\end{trivlist}}
\newenvironment{example}[1][Example]{\begin{trivlist}
\item[\hskip \labelsep {\bfseries #1}]}{\end{trivlist}}
\newenvironment{remark}[1][Remark]{\begin{trivlist}
\item[\hskip \labelsep {\bfseries #1}]}{\end{trivlist}}

\newcommand{\qed}{\nobreak \ifvmode \relax \else
      \ifdim\lastskip<1.5em \hskip-\lastskip
      \hskip1.5em plus0em minus0.5em \fi \nobreak
      \vrule height0.75em width0.5em depth0.25em\fi}

\title{Evolvement of Probability Theory}
\author{Shuai}
\date{}

\begin{document}
\maketitle

This is the text at the very beginning of the book \textit{Probability Theory} writen
by Loeve. Mathematics could be pure, but every part of science comes from
reality. Do not be confused by its abstractness and generality.

Probability theory is concerned with the mathematical analysis of the intuitive
notion of "chance" or "randomness", which, like all notions, is born of
experience. The quantitative idea of randomness first took form at the gaming
tables, and probability theory began, wich Pascal and Fermat(1654), as a theory
of games of chance. Since then, the notion of chance has found its way into
almost all branches of knowledge. In particular, the discovery that physical
"observables", even those which describe the behavior of elementary particles,
were to be considered as subject to laws of change made an inverstigation of
the notion of chance basic to the whole problem of rational interpretation of
nature.

A theory becomes mathematical when it sets up a mathematical model of the
phenomena with which it is concerned, that is, when, to describe the phenomena,
it uses a collection of well-defined symbols and operations on the symbols. As
the number of phenomena, together with their known properties, increases, the
mathematical model evolves from early crude notions upon which our intuition
was built in the direction of higher generality and abstractness.

In this manner, the inner consistency of the model of random phenomena became
doubtful, and this forced a rebuilding of the whole structure in the second
quarter of this century, starting with a formulation in terms of axioms and
definitions. Thus there appeared a branch of pure mathematics -- probability
theory -- concerned with the construction and inverstigation per se of the
mathematical model of randomness.


\end{document}
