% vim:tw=72 sw=2 ft=tex
%         File: thoughts_report.tex
% Date Created: 2013 Jun 18
%  Last Change: 2013 Dec 25
%       Author: hhiker
\documentclass[a4paper]{article}
\usepackage{xltxtra}
\usepackage{xcolor}
\usepackage{xeCJK}
\usepackage{minted}
\usepackage{booktabs}
\usepackage{amsmath}
\usepackage{paralist}
\usepackage[colorlinks=true,linkcolor=red]{hyperref}
% \usepackage{hyperref}
\usepackage{varioref}
\usepackage{cleveref}
\newcommand{\head}[1]{\textbf{#1}}
\setCJKmainfont[BoldFont=SimHei,ItalicFont=SimSun]{KaiTi}

\newtheorem{theorem}{Theorem}[section]
\newtheorem{lemma}[theorem]{Lemma}
\newtheorem{proposition}[theorem]{Proposition}
\newtheorem{corollary}[theorem]{Corollary}

\newenvironment{proof}[1][Proof]{\begin{trivlist}
\item[\hskip \labelsep {\bfseries #1}]}{\end{trivlist}}
\newenvironment{definition}[1][Definition]{\begin{trivlist}
\item[\hskip \labelsep {\bfseries #1}]}{\end{trivlist}}
\newenvironment{example}[1][Example]{\begin{trivlist}
\item[\hskip \labelsep {\bfseries #1}]}{\end{trivlist}}
\newenvironment{remark}[1][Remark]{\begin{trivlist}
\item[\hskip \labelsep {\bfseries #1}]}{\end{trivlist}}

\newcommand{\qed}{\nobreak \ifvmode \relax \else
      \ifdim\lastskip<1.5em \hskip-\lastskip
      \hskip1.5em plus0em minus0.5em \fi \nobreak
      \vrule height0.75em width0.5em depth0.25em\fi}

\title{Evolvement of Probability Theory}
\author{Shuai}
\date{}

\begin{document}
\maketitle
\tableofcontents
\pagebreak

\section{Introduction}

This is the text at the very beginning of the book \textit{Probability Theory} writen
by Loeve. Mathematics could be pure, but every part of science comes from
reality. Do not be confused by its abstractness and generality.

Probability theory is concerned with the mathematical analysis of the intuitive
notion of "chance" or "randomness", which, like all notions, is born of
experience. The quantitative idea of randomness first took form at the gaming
tables, and probability theory began, wich Pascal and Fermat(1654), as a theory
of games of chance. Since then, the notion of chance has found its way into
almost all branches of knowledge. In particular, the discovery that physical
"observables", even those which describe the behavior of elementary particles,
were to be considered as subject to laws of change made an inverstigation of
the notion of chance basic to the whole problem of rational interpretation of
nature.\cite{1977probability}

A theory becomes mathematical when it sets up a mathematical model of the
phenomena with which it is concerned, that is, when, to describe the phenomena,
it uses a collection of well-defined symbols and operations on the symbols. As
the number of phenomena, together with their known properties, increases, the
mathematical model evolves from early crude notions upon which our intuition
was built in the direction of higher generality and
abstractness.\cite{1977probability}

In this manner, the inner consistency of the model of random phenomena became
doubtful, and this forced a rebuilding of the whole structure in the second
quarter of this century, starting with a formulation in terms of axioms and
definitions. Thus there appeared a branch of pure mathematics -- probability
theory -- concerned with the construction and inverstigation per se of the
mathematical model of randomness.\cite{1977probability}

\section{Interpretations of Probability}

	Intuitively, the probability $P(\alpha)$ of an event $\alpha$
	quantifies the degree of confidence that $\alpha$ will occur. If
	$P(\alpha) = 1$, we are certain that one of the outcomes in $\alpha$
	occurs. However, this description does not provide an answer to waht
	the numbers mean. There are two common interpretations for
	probabilities.\cite{koller2009probabilistic}

	The frequentist interpretation views probabilities as frequencies of
	events. More precisely, the probability of an event is the fraction of
	times the event occurs if we repeat the experiment indefinitely. This
	interpretation gives probabilities a tangible semantics. When we
	discuss concrete physical systems(for example, dice, coin, flips, and
	card games) we can envision how these frequencies are
	defined.\cite{koller2009probabilistic}

	The frequentist interpretation fails, however, when we consider events
	such as "It will rain tomorrow afternoon." Although the time span of
	"Tomorrow afternoon" is somewhat ill defined, we expect it to occur
	exactly once. It is not clear how we define the frequencies of such
	events. Several attempts have been made to define the probability for
	such an event by finding a \textit{reference class} of similar for
	which frequencies are well defined, but they are not
	satisficatory.\cite{koller2009probabilistic}

	An alternative interpretation views probabilities as subjective
	degrees of belief. Under this interpretation, the statement $P(\alpha)
	= 0.3$ represents a subjective statement about one's own degree of
	belief that the event $\alpha$ will come
	about.\cite{koller2009probabilistic} What does it mean by saying
	subjective degrees of belief?

\section{Intuitive Background}

\subsection{Events}

The primary notion in the understanding of nature is that of
\textit{event} -- the occurrence or nonocurrence of a phenomenon. The
abstract concept of event pertains only to its occurrence or
nonocurrence and not to its nature. This is the concept we intend to
analyze.\cite{1977probability}

In science, or, more precisely, in the investigation of ``laws of
nature,'' events are classified into conditions and outcomes of an
experiment. \textit{Conditions} of an experiment are events which are known or
are made to occur. \textit{Outcomes} of an experiment are events which
\textit{may} occur when the experiment is performed, that is, when its
conditions occur. All(finite) combinations of outcomes by means of
``not'', ``and'', ``or'', are outcomes; in the terminology of sets, the
outcomes of an experiment form a \textit{field}(or an ``algebra'' of
sets). The condition of an experiment together with its field of
outcomes, constitute a \textit{trial}. Any (finite) number of trials can
be combined by ``conditioning'', as following:

The collective outcomes are combinations by means of ``not'', ``and'',
``or'', of the outcomes of the constituent trials. The conditions are
conditions of the first constituent trials together with conditions of
the second to which are added the observed outcomes of the first, and so
on. Thus, given the observed outcomes the preceding trials, every
constituent trial is performed under supplementary conditions: it is
conditioned by the observed outcomes. When, for every constituent trial,
any outcome occurs if, and only if, it occurs without such conditioning,
we say that the trials are \textit{completely independent}. If,
moreover, the trials are identical, that is, have the same conditions
and the same field of outcomes, we speak of \textit{repeated trials} or
equivalently,, \textit{identical and completely independent trials}. The
possibility of repeated trials is a basic assumption in science, and in
games of chance: \textit{evry trial can be performed again and again,
the knowledge of past and present outcomes having no influence upon
future ones.}\cite{1977probability}

\section{Where I Stop}
	Initially, I expect to find a answer to the question ``What does it
	mean by saying subjective degrees of belief?'' by reading Intuitive
	Background Chapter of the book\cite{1977probability}. However, after
	some quick check, no easy question is provided. Thus, this doc stops
	here for the time being.





\bibliographystyle{plain}
\bibliography{../../reference_lib/reference}

\end{document}
