% vim:tw=72 sw=2 ft=tex
%         File: thoughts_report.tex
% Date Created: 2013 Jun 18
%  Last Change: 2013 Dec 16
%       Author: hhiker
\documentclass[a4paper]{article}
\usepackage{xltxtra}
\usepackage{xcolor}
\usepackage{xeCJK}
\usepackage{minted}
\usepackage{booktabs}
\usepackage{amsmath}
\usepackage{paralist}
\usepackage[colorlinks=true,linkcolor=red]{hyperref}
% \usepackage{hyperref}
\usepackage{varioref}
\usepackage{cleveref}
\newcommand{\head}[1]{\textbf{#1}}
\setCJKmainfont[BoldFont=SimHei,ItalicFont=SimSun]{KaiTi}

\newtheorem{theorem}{Theorem}[section]
\newtheorem{lemma}[theorem]{Lemma}
\newtheorem{proposition}[theorem]{Proposition}
\newtheorem{corollary}[theorem]{Corollary}

\newenvironment{proof}[1][Proof]{\begin{trivlist}
\item[\hskip \labelsep {\bfseries #1}]}{\end{trivlist}}
\newenvironment{definition}[1][Definition]{\begin{trivlist}
\item[\hskip \labelsep {\bfseries #1}]}{\end{trivlist}}
\newenvironment{example}[1][Example]{\begin{trivlist}
\item[\hskip \labelsep {\bfseries #1}]}{\end{trivlist}}
\newenvironment{remark}[1][Remark]{\begin{trivlist}
\item[\hskip \labelsep {\bfseries #1}]}{\end{trivlist}}

\newcommand{\qed}{\nobreak \ifvmode \relax \else
      \ifdim\lastskip<1.5em \hskip-\lastskip
      \hskip1.5em plus0em minus0.5em \fi \nobreak
      \vrule height0.75em width0.5em depth0.25em\fi}

\title{Development of Algebra}
\author{Shuai}
\date{}

\begin{document}
\maketitle

Algebra can essentially be considered as doing computations similar to
that of arithmetic with non-numerical mathematical objects.
Initially, these objects represented either numbers that were not yet
known (unknowns) or unspecified numbers (indeterminates or parameters),
allowing one to state and prove properties that are true no matter which
numbers are involved. For example, in the quadratic equation

\begin{displaymath}
	ax^2+bx+c=0
\end{displaymath}

$a$, $b$, $c$ are indeterminates and $x$ is the unknown. Solving this equation
amounts to computing with the variables to express the unknown x in
terms of the indeterminates. Then, substituting any numbers for the
indeterminates, gives the solution of a particular equation after a
simple arithmetic computation.

As it developed, algebra was extended to other non-numerical objects,
like vectors, matrices or polynomials. Then, the structural properties
of these non-numerical objects were abstracted to define algebraic
structures like groups, rings, fields and algebras.

Before the 16th century, mathematics was divided into only two
subfields, arithmetic and geometry. Even though some methods, which had
been developed much earlier, may be considered nowadays as algebra, the
emergence of algebra and, soon thereafter, of infinitesimal calculus as
subfields of mathematics only dates from 16th or 17th century. From the
second half of 19th century on, many new fields of mathematics appeared,
some of them included in algebra, either totally or partially.

It follows that algebra, instead of being a true branch of mathematics,
appears nowadays, to be a collection of branches sharing common methods.
This is clearly seen in the Mathematics Subject Classification where
none of the first level areas (two digit entries) is called algebra. In
fact, algebra is, roughly speaking, the union of sections 08-General
algebraic systems, 12-Field theory and polynomials, 13-Commutative
algebra, 15-Linear and multilinear algebra; matrix theory,
16-Associative rings and algebras, 17-Nonassociative rings and algebras,
18-Category theory; homological algebra, 19-K-theory and 20-Group
theory. Some other first level areas may be considered to belong
partially to algebra, like 11-Number theory (mainly for algebraic number
theory) and 14-Algebraic geometry.

Elementary algebra is the part of algebra that is usually taught in
elementary courses of mathematics.

Abstract algebra is a name usually given to the study of the algebraic
structures themselves.

from: \href{http://en.wikipedia.org/wiki/Algebra}{wikipedia}

\end{document}
