% vim:tw=72 sw=2 ft=tex
%         File: thoughts_report.tex
% Date Created: 2013 Jun 18
%  Last Change: 2013 Dec 24
%       Author: hhiker
\documentclass[a4paper]{article}
\usepackage{xltxtra}
\usepackage{xcolor}
\usepackage{xeCJK}
\usepackage{minted}
\usepackage{booktabs}
\usepackage{amsmath}
\usepackage{paralist}
\usepackage[colorlinks=true,linkcolor=red]{hyperref}
% \usepackage{hyperref}
\usepackage{varioref}
\usepackage{cleveref}
\newcommand{\head}[1]{\textbf{#1}}
\setCJKmainfont[BoldFont=SimHei,ItalicFont=SimSun]{KaiTi}

\newtheorem{theorem}{Theorem}[section]
\newtheorem{lemma}[theorem]{Lemma}
\newtheorem{proposition}[theorem]{Proposition}
\newtheorem{corollary}[theorem]{Corollary}
\newtheorem{definition}{Definition}[section]
\newtheorem{remark}{Remark}[section]

\newenvironment{proof}[1][Proof]{\begin{trivlist}
\item[\hskip \labelsep {\bfseries #1}]}{\end{trivlist}}
\newenvironment{example}[1][Example]{\begin{trivlist}
\item[\hskip \labelsep {\bfseries #1}]}{\end{trivlist}}

\newcommand{\qed}{\nobreak \ifvmode \relax \else
      \ifdim\lastskip<1.5em \hskip-\lastskip
      \hskip1.5em plus0em minus0.5em \fi \nobreak
      \vrule height0.75em width0.5em depth0.25em\fi}

\title{The Natural Numbers}
\author{Shuai}


\begin{document}
\maketitle

\section{Axiomization of the intuition -- Peano Axiom}

	The notion of a natural number is one of the most fundamental and most
	important in mathematics. The system of natural numbers was the first
	abstract scientific concept created by man. Having dealt in everyday
	life, with certain quantitieso of real things, people noted certain
	general properties of numbers and developed the notion of counting
	numbers. This apparently simple concept is in some ways so profound that
	it has prompted some people to believe that this concept comes directly
	from God. A great German number theorist, Leopold Kronecker(1823 - 1891)
	said:"God made the natural numbers, all else is the work of
	man."[Heinrich Weber. Leopold Kronecker. Jahresberichte DMV 1893;
	2:5-31]. Creating the notion of a natural number is first step not only
	in mathematics, but in the development of all
	sciences\cite{dixon2011algebra}.

	The history is long, however, the modern axiomatic theory of natural
	numbers is developed at the end of the nineteenth century and named in
	honor of a famous Italian mathematician, Giuseppe Peano(1858 - 1932),
	whose input in the axiomatization of natural numbers was of exceptional
	mathematical and philosopical value\cite{dixon2011algebra}.

	Mathematics comes from everyday life, but abstracts it, then provides a
	solid foundation for further reasoning, development of more advanced
	theory.

	The peano axiom is the very foundation of number theory, or in some
	sense modern mathematics. It clearly defines what we used to count in
	real life, meaning axiomize it.

	\begin{definition}
		\textbf{Peano Axiom}: The set $N_0$ is a nonempty set and for all $ a
		\in N_0 $, there is a uniquely defined element $a'$, called the
		immediate successor of a and for which the following axioms hold:
		\begin{itemize}
			\item
				\textbf{(P 1)}  a = b implies that $a' = b'$.
			\item
				\textbf{(P 2)}  There is an element 0(the natural number 0) such
				that 0 is not the immediate successor of any element of $N_0$.
				Thus $0 \neq a'$ for all elements $ a \in N_0 $.
			\item
				\textbf{(P 3)}  If $a,b \in N_0$ and $ a' = b'$, then $a = b$.
			\item
				\textbf{(P 4)}  (the induction axiom). Let M be a subset of $N_0$
				satisfying the conditions:
				\begin{itemize}
					\item $ 0 \in M$;
					\item if $a \in M$, then $a' \in M$.
				\end{itemize}
				Then $ M = N_0 $.
		\end{itemize}
	\end{definition}

	The natural number is defined as\cite{dixon2011algebra}:

	\begin{displaymath}
		0 = \emptyset, 1 = \{\emptyset\}, 2 = 1 \cup \{1\} = \{0, 1\}, 3 = 2 \cup
		\{2\} = \{0, 1, 2\} ...
	\end{displaymath}

	In the book\cite{dixon2011algebra}, the author does not talk about why
	natural number is defined this way:
	\begin{quote}
		Such a level of exposition is far beyond the scope of this book and
		requires significant mathematical maturity.
	\end{quote}

	I just take those definition as natural abstraction of the counting
	symbol used by people.

	With the Peano Axiom, all normal arithmetical properties can be derived.

	Ok. I think this is where I should stop digging.

\bibliographystyle{plain}
\bibliography{../reference_lib/reference}

\end{document}
